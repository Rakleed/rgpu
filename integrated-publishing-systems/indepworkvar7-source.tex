% "Вариативная самостоятельная работа № 7"

\documentclass[a4paper,12pt]{article} % тип документа -- лсит А4 с 12 шрифтом текста

\usepackage[T2A]{fontenc}			% кодировка
\usepackage[utf8]{inputenc}			% кодировка исходного текста
\usepackage[english,russian]{babel}	% локализация и переносы

\usepackage{amsmath,amsfonts,amssymb,amsthm,mathtools} % математические символы, формулы и т. д.

\usepackage{wasysym}

\usepackage{hyperref}

\author{Моисеенко П. А., 1 гр. 2 подгр.} % имя автора
\title{Таблица команд} % название документа
\date{\today} % дата создани документа

\begin{document}
\maketitle
\newpage
\section{Формулы сокращённого умножения в \LaTeX}
\subsection{Разность квадратов}
\begin{equation}
a^2-b^2=(a-b)*(a+b)
\end{equation}

\subsection{Квадрат суммы двух чисел}
\begin{equation}
(a+b)^2=a^2+2*a*b+b^2
\end{equation}

\subsection{Квадрат разности двух чисел}
\begin{equation}
(a-b)^2=a^2-2*a*b+b^2
\end{equation}

\subsection{Сумма кубов}
\begin{equation}
a^3+b^3=(a+b)*(a^2-a*b+b^2)
\end{equation}

\subsection{Разность кубов}
\begin{equation}
a^3-b^3=(a-b)*(a^2+a*b+b^2)
\end{equation}

\subsection{Куб суммы двух чисел}
\begin{equation}
(a+b)^3=a^3+3*a^2*b+3*a*b^2+b^3
\end{equation}

\subsection{Куб разности двух чисел}
\begin{equation}
(a-b)^3=a^3-3*a^2*b+3*a*b^2-b^3
\end{equation}

\end{document}
