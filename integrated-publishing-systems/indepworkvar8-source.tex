% "Вариативная самостоятельная работа № 8"

\documentclass[a4paper,12pt]{article} % тип документа -- лсит А4 с 12 шрифтом текста

\usepackage[T2A]{fontenc}			% кодировка
\usepackage[utf8]{inputenc}			% кодировка исходного текста
\usepackage[english,russian]{babel}	% локализация и переносы

\usepackage{amsmath,amsfonts,amssymb,amsthm,mathtools} % математические символы, формулы и т. д.

\usepackage{wasysym}

\usepackage{hyperref}

\author{Моисеенко П. А., 1 гр. 2 подгр.} % имя автора
\title{Создание матриц средствами \LaTeX} % название документа
\date{\today} % дата создани документа

\begin{document}
\maketitle
\newpage
\section{Матрицы}
\subsection{Задание 1}
\textbf{Дано:}
$$ A=\begin{pmatrix}
1& 2& 3 \\
4& 5& 6
\end{pmatrix} $$
\textbf{Решение:}
$$ B=2\times A=2\times \begin{pmatrix}
1& 2& 3 \\
4& 5& 6
\end{pmatrix}=
\begin{pmatrix}
2& 4& 6 \\
8& 10& 12
\end{pmatrix} $$
\textbf{Ответ:}
$$ B=\begin{pmatrix}
2& 4& 6 \\
8& 10& 12
\end{pmatrix} $$

\subsection{Задание 3}
\textbf{Дано:}
$$ A=\begin{pmatrix}
7& 8& 9 \\
1& 2& 3
\end{pmatrix} $$
\textbf{Решение:}
$$ A=\begin{pmatrix}
7& 8& 9 \\
1& 2& 3
\end{pmatrix} 
\Rightarrow A^T=
\begin{pmatrix}
7& 1 \\
8& 2 \\
9& 3
\end{pmatrix} $$
\textbf{Ответ:}
$$ A^T=\begin{pmatrix}
7& 1 \\
8& 2 \\
9& 3
\end{pmatrix} $$
\end{document}
