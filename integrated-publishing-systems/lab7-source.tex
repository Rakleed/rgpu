% "Лабораторная работа № 7"

\documentclass[a4paper,12pt]{article} % тип документа

% report, book

% Русский язык

\usepackage[T2A]{fontenc}			% кодировка
\usepackage[utf8]{inputenc}			% кодировка исходного текста
\usepackage[english,russian]{babel}	% локализация и переносы


% Математика
\usepackage{amsmath,amsfonts,amssymb,amsthm,mathtools} 


\usepackage{wasysym}

\usepackage{hyperref}

%Заговолок
\author{Моисеенко П. А., 1 гр. 2 подгр.}
\title{Особенности технологии создания текста с формулами в \LaTeX{}}
\date{\today}


\begin{document} % начало документа
\maketitle
\newpage
\section{Формулы}
\subsection{Встраиваемая (включенная) формула}
Площадь прямоугольника определяется по формуле $ S=ab $ известной из школьного курса математики. Например, $ 2+2=4 $ называется равенством.

\subsection{Выключенная формула}
Формула по центру строки $$ 1+3=4 $$
Теорема Пифагора $$ a^2+b^2=c^2 $$ часто применяется при решении различных геометрических задач.

\subsection{Нумерация формул}
\begin{equation}
a+b=b+a
\end{equation}
\begin{equation} \label{pifagor}
a^2+b^2=c^2
\end{equation}

Чтобы сослаться на формулу, которая стоит в тексте намног ораньше, можно использовать команду \textbf{eqref}\\
Например.\\
Как было сказано раньше в
\eqref{pifagor} гипотенуза определена.
Об этом было уже сказано на странице \pageref{pifagor}.

\section{Дроби}
$\frac{1}{4}+\frac{1}{4}=\frac{2}{4}=\frac{1}{2}$ это больше по высоте, чем текст. Чтобы не изменять внешний вид текста используют выключные формулы.\\

Поэтому в случае использования обыкновенных дробей используйте выключные формулы.
$$\frac{1}{4}+\frac{1}{4}=\frac{2}{4}=\frac{1}{2}$$

\section{Скобки}
$$ (2+3)*5=25 $$
$$ (2+3) \times 5=25 $$
$$ (2+3) \cdot 5=25 $$
\subsection*{Размер скобок}
$$ (\frac{4}{2}+3) \cdot 5=25 $$
$$ \left(\frac{4}{2}+3\right) \cdot 5=25 $$ Размер подбирается автоматически для любых скобок при использовании \textbf{left и right}.
$$ \{2+3\} \cdot 5=25 $$

\section{Индексы и показатели}
$m_1$\\
$c^2$

Если аргумент состоит из более чем одного символа, то его следует взять в фигурные скобки.\\
$m_{11}$\\
$c^{22}$

\section{Стандартные функции}
$\sin x=0$\\
$\arctan x=\sqrt{3}$\\
$\arcctg a=\sqrt[5]{3}$\\
$\log_{x-1}{(x^2+3x-4)} \geqslant2$\\
$\lg x=\ln a$\\
$\sum_{x=1}^{n}a_i+b_j$
\begin{equation}
\sum_{x=1}^{n}a_i+b_j
\end{equation}

\subsection*{Интеграл}
$I=\int r^2dm$\\
$I=\int _{0}^{1} r^2dm$\\
$$I=\int\limits _{0}^{1} r^2dm$$\\
\href{http://detexify.kirelabs.org/classify.html}{Найти код символа}\\
Львовский Набор и верстка в системе \LaTeX\\
Пособие стр. 45--72 (ссылка ниже в тексте)\\
\href{https://bit.ly/3mvbiuv}{Пособие}\\

$$ \int \frac{dx}{\ln x}=\ln |\ln x| + \sum_{i-1}^{\infty}\frac{(\ln x)^i}{i\cdot i!} $$
$$ \int \frac{(\ln x)^n*dx}{x}=\frac{(\ln x)^{n+1}}{n+1} \text{ для } n\neq -1 $$

\end{document}
