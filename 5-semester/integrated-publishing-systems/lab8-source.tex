% "Лабораторная работа № 8"

\documentclass[a4paper,12pt]{article} % тип документа

% report, book

% Русский язык

\usepackage[T2A]{fontenc}			% кодировка
\usepackage[utf8]{inputenc}			% кодировка исходного текста
\usepackage[english,russian]{babel}	% локализация и переносы


% Математика
\usepackage{amsmath,amsfonts,amssymb,amsthm,mathtools} 


\usepackage{wasysym}

\usepackage{hyperref}

%Заговолок
\author{Моисеенко П. А., 1 гр. 2 подгр.}
\title{Создание матриц средствами \LaTeX}
\date{\today}


\begin{document}
\maketitle
\newpage
\section{Диактрические знаки}
\subsection{Надстрочные}
$$ \dot{x}=0 $$\\
$$ \tilde{a}=\bar{b} $$
$$ \tilde{a}=\overline{bcde} $$
широкая тильда\\
$$ \widetilde{afgh}=\overline{bcde} $$

многоточие $\cdots$

\subsection{Веткторы}
Вектор a имеет координаты (0;3;4)

$$ \overrightarrow{a}(0;3;4) $$

Запись вектора жирным шрифтом, а не стрелкой сверху
$$ \overrightarrow{a}=\mathbf{a} $$

\subsection{Фигурная скобка}
$$ \underbrace{1+2+\cdots+n}=N $$

$$ \underbrace{1+2+\cdots+n}_{n}=N $$

\begin{equation}
\underbrace{1+2+\cdots+n}=N
\end{equation}

\begin{equation}
\underbrace{1+2+\cdots+n}_{n}=N
\end{equation}

\begin{equation}
\overbrace{1+2+\cdots+n}^{n}=N
\end{equation}

\subsection{Написание условия перехода над знаком}
команда \textbf{stackrel}\\
Например.\\
$$ (x-1)(x+1)>0 \stackrel{x>0}{\longleftrightarrow}(x-1)>0 $$

\subsection{Буквы других алфавитов}
$$ \sin \alpha=0 $$
$$ \omega=\frac{2\pi}{T} $$
непривычный вид\\
$\epsilon$ \\
$\phi$\\\\
как в учебнике\\
$\varepsilon$\\
$\varphi$

\subsection{Математические шрифты}
много\\
один из них $\mathbf{mathbb}$
находится во вкладке Математика/Математические шрифты\\
$$ x \in R $$
$$ x \in \mathbb{R} $$

\subsection{Кириллические символы}
используется команда \textbf{text}
$$ m_{\text{груза}}=15~{\text{кг}} $$
Для пробела между обозначениями величины и её численными значением необходимо использовать тильду

\section{Выравнивание формул}
окружение \textbf{aligned}\\
определяет выравнивание амперсант \&

\section{Группировка формул}
\begin{equation}
\begin{aligned}
4&\times a=8 \\
-5&\times b=10 \\
-10&\times c=110 \\
\end{aligned}
\end{equation}
 
\subsection{Системы уравнений}
$$ \left \{
\begin{aligned}
4&\times a=8 \\
-5&\times b=10 \\
-10&\times c=110 \\
\end{aligned} \right. $$

$$ \left.
\begin{aligned}
4&\times a=8 \\
-5&\times b=10 \\
-10&\times c=110 \\
\end{aligned} \right \} $$

$$ \left.
\begin{aligned}
4&\times a=8 \\
-5&\times b=10 \\
-10&\times c=110 \\
\end{aligned} \right \} \Rightarrow -12ab=24 $$

\section{Матрицы}
Создаются за счёт окружения \textbf{matrix}

\subsection{Матрица в круглых скобках}
$$ \begin{pmatrix}
a_{11}& a_{12}& a_{13} \\
a_{21}& a_{22}& a_{23} \\
a_{31}& a_{32}& a_{33}
\end{pmatrix} $$

\subsection{Матрица в квадратных скобках}
$$ \begin{bmatrix}
a_{11}& a_{12}& a_{13} \\
a_{21}& a_{22}& a_{23} \\
a_{31}& a_{32}& a_{33}
\end{bmatrix} $$

\subsection{Определитель}

$$ \begin{Vmatrix}
a_{11}& a_{12}& a_{13} \\
a_{21}& a_{22}& a_{23} \\
a_{31}& a_{32}& a_{33}
\end{Vmatrix} $$

$$ \begin{vmatrix}
a_{11}& a_{12}& a_{13} \\
a_{21}& a_{22}& a_{23} \\
a_{31}& a_{32}& a_{33}
\end{vmatrix} $$

\end{document}
