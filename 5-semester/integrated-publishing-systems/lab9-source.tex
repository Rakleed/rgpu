% "Лабораторная работа № 9"

\documentclass[a4paper,12pt]{article} % тип документа

% report, book

% Русский язык

\usepackage[T2A]{fontenc}			% кодировка
\usepackage[utf8]{inputenc}			% кодировка исходного текста
\usepackage[english,russian]{babel}	% локализация и переносы


% Математика
\usepackage{amsmath,amsfonts,amssymb,amsthm,mathtools} 


\usepackage{wasysym}

\usepackage{hyperref}

%Заговолок
\author{Моисеенко П. А., 1 гр. 2 подгр.}
\title{Создание таблиц в \LaTeX}
\date{\today}


\begin{document}
\maketitle
\newpage
\section{Создание таблицы}
\subsection{Инструменты}
\begin{enumerate}
\item Вкладка Помощник
\item Быстрая таблица
\end{enumerate}

или

\begin{enumerate}
\item LaTeX
\item Таблицы
\end{enumerate}

\subsection{Быстрая таблица}
\subsubsection{Шаг 1}
Выберите количество столбцов и количество строк\\
\begin{tabular}{|c|c|c|c|c|}
\hline
• & • & • & • & • \\
\hline
• & • & • & • & • \\
\hline
• & • & • & • & • \\
\hline
• & • & • & • & • \\
\hline
• & • & • & • & • \\
\hline
\end{tabular}

\subsubsection{Шаг 2}
Введите текст в таблицу\\
\begin{tabular}{|c|c|c|c|c|}
\hline
№ & Критерии & Параметры & • & • \\
\hline
• & • & выполнено полностью & выполнено частично & не выполнено \\
\hline
1 & • & • & • & • \\
\hline
2 & • & • & • & • \\
\hline
3 & • & • & • & • \\
\hline
\end{tabular}

\subsubsection{Шаг 3}
Объедините ячейки\\
\begin{tabular}{|c|c|c|c|c|}
\hline
№ & Критерии & \multicolumn{3}{|c|}{Параметры} \\
\hline
• & • & выполнено полностью & выполнено частично & не выполнено \\
\hline
1 & • & • & • & • \\
\hline
2 & • & • & • & • \\
\hline
3 & • & • & • & • \\
\hline
\end{tabular}

\end{document}
