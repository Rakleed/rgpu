% "Лабораторная работа № 6-2"

\documentclass[a4paper,12pt]{article} % тип документа

% report, book

% Русский язык

\usepackage[T2A]{fontenc}			% кодировка
\usepackage[utf8]{inputenc}			% кодировка исходного текста
\usepackage[english,russian]{babel}	% локализация и переносы


% Математика
\usepackage{amsmath,amsfonts,amssymb,amsthm,mathtools} 


\usepackage{wasysym}

\usepackage{hyperref}

%Заговолок
\author{Моисеенко П. А., 1 гр. 2 подгр.}
\title{Особености технологии набора технического текста в \LaTeX{}}
\date{\today}


\begin{document} % начало документа
\maketitle
\newpage
\section*{Аннотированный список ресурсов интернета, содержащих рекомендации по работе в \LaTeX}
\begin{enumerate}
\item \href{https://losst.ru/kak-polzovatsya-latex}{<<Как пользоваться LaTeX>>} --- инструкция по установке и началу использования \LaTeX. Знакомство с интерфейсом приложения и добавление формул.
\item \href{https://www.ibm.com/developerworks/ru/library/latex_tutorial_01/index.html}{<<Работа в LaTeX. Создание документа на примере подготовки курсовой работы>>} --- знакомство с \LaTeX для создания курсовой. Вставка картинок, формул, кода.
\item \href{https://ru.coursera.org/learn/latex}{<<Документы и презентации в LaTeX>>} --- курс от Coursera и ВШЭ по тому, как использовать \LaTeX в создании документов и презентаций.
\item \href{http://scibooks.narod.ru/ladocs/begin.html}{<<Начало работы в LaTeX
>>} --- самые основы работы в \LaTeX.
\item \href{https://en.wikibooks.org/wiki/LaTeX/Basics}{<<LaTeX/Basics>>} --- справочник по командам \LaTeX для начинающих пользователей. А также информация по комментариям и названию файлов.
\end{enumerate}
\end{document}
