% "Инвариативная самостоятельная работа № 8"

\documentclass[a4paper,12pt]{article} % тип документа -- лсит А4 с 12 шрифтом текста

\usepackage[T2A]{fontenc}			% кодировка
\usepackage[utf8]{inputenc}			% кодировка исходного текста
\usepackage[english,russian]{babel}	% локализация и переносы

\usepackage{amsmath,amsfonts,amssymb,amsthm,mathtools} % математические символы, формулы и т. д.

\usepackage{wasysym}

\usepackage{hyperref}

\author{Моисеенко П. А., 1 гр. 2 подгр.} % имя автора
\title{Таблица команд} % название документа
\date{\today} % дата создани документа

\begin{document}
\maketitle
\newpage
\section*{Команды для набора матриц в \LaTeX}
\begin{tabular}{ l | r }
  \textbf{Назначение команды}	& \textbf{Вид (написание) команды} \\ \hline
  точка над символом			& dot \\ \hline
  тильда над символом			& tilde \\ \hline
  черта над символами			& overline \\ \hline
  широкая тильда				& wildtilde \\ \hline
  многоточие					& cdots \\ \hline
  вектор						& overrightarrow \\ \hline
  математический символ, жирный	& mathbf \\ \hline
  фигурная скобка снизу			& underbrace \\ \hline
  фигурная скобка сверху		& overbrace \\ \hline
  условие над знаком			& stackrel \\ \hline
  стрелка влево-вправо			& longleftrightarrow \\ \hline
  альфа							& alpha \\ \hline
  омега							& omega \\ \hline
  пси							& varpsi \\ \hline
  эпсилон						& varepsilon \\ \hline
  вхождение						& in \\ \hline
  выравнивание					& begin {align}, end{align} \\ \hline
  фантомная скобка				& left., right. \\ \hline
  матрица в круглых скобках		& begin {pmatrix}, end{pmatrix} \\ \hline
  матрица в квадратных скобках	& begin {bmatrix}, end{bmatrix} \\ \hline
  определитель матрицы			& begin {vmatrix}, end{vmatrix}
\end{tabular}
\end{document}
